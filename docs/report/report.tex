\documentclass[UTF8]{article}

\usepackage{ctex}
\usepackage{fancyhdr} %自定义页眉页脚
\usepackage{listings} %source code
\usepackage{tabu}
\usepackage{booktabs}
\usepackage{graphicx}
\usepackage[colorlinks,linkcolor=blue]{hyperref}
\usepackage{multirow}

\begin{document}


\begin{titlepage}
\center{北京理工大学计算机学院2017级}
\vspace{2cm}
\center{\huge{《数字逻辑》实验二  实验报告}}
\vspace{0.5cm}
\center{\Huge{一个三开关的低电压照明系统}}
\vspace{2cm}

\begin{center}
\begin{large}
\begin{tabular}{c c |c c}
小组编号& 44 & 指导老师 & 王娟\\
\cline{2-2} \cline{4-4}
\hline
学\qquad 号& 1120173323 & 电\qquad 话& 592175501@qq.com \\
\cline{2-2} \cline{4-4}
姓\qquad 名& 王聚海 & 邮\qquad 箱& 13521592752 \\
\cline{2-2} \cline{4-4}
班\qquad 级 & 07111703 \\
\cline{2-2}
\hline
学\qquad 号& 1120171224 \\
\cline{2-2}
姓\qquad 名& 冯开宇 \\
\cline{2-2}
班\qquad 级 & 07111701 \\
\cline{2-2}
\hline
学\qquad 号& 1120172200 \\
\cline{2-2}
姓\qquad 名& 刘思雨 \\
\cline{2-2}
班\qquad 级 & 07111701 \\
\cline{2-2}



\end{tabular}
\end{large}
\end{center}
\vfill \hfill
\end{titlepage}
\clearpage


\section{设计题目}

\begin{center}
    
\center{一个三开关的低电压的照明系统}
\end{center}

\paragraph{}
一个低电压的照明系统采用二进制逻辑控制器控制一盏特殊灯的照明,
这盏灯用于T型走廊的交叉口。
在T型走廊的三个端点各有一个控制灯的开关,
这些开关的开合状态决定它们的二进制输出是0还是1,
三个开关分别用$X1$、$X2$和$X3$表示。
这盏灯由一个带缓冲驱动的可控硅控制,
可控硅控制电灯电源电路的导通。当缓冲器的输入$Z$为1的时候,
灯是开着的,当$Z$为0的时候,灯是熄灭的。
你需要得到这样一个函数$Z = F(X1, X2, X3)$,如果任意一个开关变化了,
$Z$的值就会改变,从而控制电灯的开和关。

\section{设计目的}
\begin{enumerate}
    \item 利用组合逻辑电路,按照题目要求,设计电路模块,
    \item 利用三个开关$X1,X2,X3$的输出控制缓冲器输入$Z$,
    \item 从而控制一个灯的亮与灭,并结合开发板进行演示验证。
\end{enumerate}

\section{设备器材}
\begin{itemize}
    \item Xilinx Vivado开发环境
    \item EES-338口袋计算机硬件平台,配备 FPGA (XC7A35TCSG324-1C)

\end{itemize}

\section{设计原理与内容}
\subsection{问题分析}

我们可以由分析得出,输出$Z$同时收$X1,X2,X3$影响,
且当任意一个值发生变化时,Z取反。
我们很快能够联想到格雷码的内容——格雷码就是相邻状态只有一位的改变。

该系统的初始状态为:$X1, X2, X3$都处于关闭状态,故它们的输出都是0。
初始时灯是不亮的,故缓冲器的输入$Z$也是0。
之后,只要任意一个开关的状态改变一次,$Z$的值就会取反。

这就意味着,只要$X1, X2, X3$中输出为1的个数是奇数,
开关状态的改变次数就是奇数,$Z$的值就是1。
这是因为,对于一个开关Xi, 若它的输出是0,则它的状态变化次数为 $2ki$;
若它的输出是1,则它的状态变化次数为$1 + 2ki$。
那么,开关的变化次数,即Z的取反次数为 $2 (k1 + k2 + k3 ) + t$,
其中$ki$为整数,$t$为输出为1的开关个数。
这样一来,该问题就成为了一个偶校验问题。


按照如上分析,可以直接列出真值表,输入按照格雷码排列,则输出从0开始不断取反。
真值表如下:

\begin{table}[!htbp]
\centering
\begin{tabu} to 0.8\textwidth{ X[c] | X[c] | X[c] | X[c] }
    \hline
    X1 & X2 & X3 & Z \\
    \hline\hline
    0 & 0 & 0 & 0\\
    0 & 0 & 1 & 1\\
    0 & 1 & 1 & 0\\
    0 & 1 & 0 & 1\\
    1 & 1 & 0 & 0\\
    1 & 1 & 1 & 1\\
    1 & 0 & 1 & 0\\
    1 & 0 & 0 & 1\\

    \end{tabu}
\caption{真值表}
\end{table}
该函数方程即为  

$$Z = X1\oplus X2\oplus X3$$

接下来就可以开始进行组合逻辑电路的设计。

\section{设计步骤}
\subsection{描述需求}

通过开关$X1$, $X2$, $X3$的状态变化,使$Z = X1⊕X2⊕X3$的值变化,
从而使灯亮或灭。

\subsection{定义输入与输出}

定义输入为长度为三位的向量 

\lstset{language=verilog}
\begin{lstlisting}[frame=tb]{somecode}
    [2:0] data_in
\end{lstlisting}

{test}
则\lstinline{data_in[0]},
\lstinline$data_in[1]$ , \lstinline$data_in[2]$分别为开关$X1$, $X2$, $X3$的输出值。
定义输出为\lstinline$data_out$,
其值即为Z的值。

\subsection{程序实现}

在模块streetlight中,使用数据流描述方式,采用\lstinline{assign}对输出进行赋值。可以直接使

用位运算符$\wedge$进行运算,即
\lstset{language=verilog}
\begin{lstlisting}[frame=tb]{somecode}
    data_out = data_in[0]^data_in[1]^data_in[2]
\end{lstlisting}

在测试模块\lstinline{streetlight_tb}中,
定义reg型的变量\lstinline{[2:0] data_in}作为输入端口,
定义wire型的变量\lstinline{data_out}作为输出端口。
然后,将模块streetlight实例化为pj2,
并指定相应的输入和输出。
在initial块中,将0-7的三位二进制数按照格雷码的顺序依次赋值给\lstinline{data_in},
每次赋值之间延迟100个时间单位。

进行RTL分析后,生成RTL原理图:
​
\begin{figure}[htbp]
    \centering
    \includegraphics[scale=0.7]{0.png}
    \caption{RTL原理图}
\end{figure}

\subsection{仿真与测试}
进行行为仿真。得到的波形与预计一致。
\subsection{运行}

\section{遇到的问题及解决方法}
无
\section{收获与体会}

在本次实验的设计中,我们最大大的收获是,将实际的问题需求描述与所学的设
计方方法相联系,化抽象问题为具体函数关系,一一步步将问题化为易易于设计、易易于编程
实现的形式。

这一一点在本实验中具体体现在,将“任意开关变化电灯开关变化”通过分析与证
明映射到偶校验问题上,使之成为我们熟悉的问题。

在问题解决过程中,我们通过分析来正确应用用数字逻辑理理论部分学习成果的能
力力力得到了了提升。

\section{小组成员分工}
刘思雨
\begin{itemize}
    \item 问题思路分析、函数构造、测试
    \item 实验报告撰写
\end{itemize}

冯开宇
\begin{itemize}
    \item 仿真测试代码编写
    \item 实验报告排版
\end{itemize}

王聚海
\begin{itemize}
    \item 程序结构设计
\end{itemize}
\section{参考资料}
无

\section{项目地址}

\href{https://github.com/bit-logic-computer-design-2019/street-light}{street-light}

\end{document}
